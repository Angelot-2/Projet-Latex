\documentclass[a4paper,11pt]{article} % ce document est un article sur une feuille A4, police taille 11

\usepackage[utf8]{inputenc} % encodé en utf-8
\usepackage[T1]{fontenc} % compatible avec les accents

\usepackage[round]{natbib} % gestion des citations
\usepackage[french]{babel} % rédigé en français
\usepackage[hyphens]{url} % formatte les liens en autorisant la césure au niveau des traits d'union
\usepackage[pdftex,urlcolor=black,colorlinks=true,linkcolor=black,citecolor=black]{hyperref} % liens cliquables mais non colorés
\usepackage[top=3cm,bottom=4cm]{geometry} % gère les marges
\usepackage{graphicx} % gestion des images
\usepackage{array} % gestion des tableaux
\usepackage{csquotes} % gestion des guillemets
\usepackage{fourier} % utilise une autre police que celle par défaut (Computer Modern)


% insérez ici d'autres extensions avec la commande \usepackage[options]{nom de l'extension}

\title{L’utilisation des réseaux sociaux dans la promotion des ressources
numériques en bibliothèque universitaire} % le titre de l'article
\author{Angelot Ngoufack Nanfah} % vos prénom et nom
\date{} % pas de date

\begin{document} % début du corps du texte \newpage
\maketitle % affiche le titre, l'auteur et la date
\newpage

\section{Introduction} % section 1 
Les bibliothèques universitaires (BU) ont dû adapter leurs méthodes de communication
avec l’arrivée et la progression rapide des réseaux sociaux (RSN) au sein de la communauté 
universitaire. Ceux-ci sont utilisés habituellement pour la promotion des services de la 
bibliothèque et de leurs ressources. En effet, les ressources numériques sont encore trop souvent 
invisibles auprès des utilisateurs des bibliothèques universitaires, dès lors cela pose un grand 
souci. Les réseaux sociaux sont des plateformes 
qui rassemblent de grandes communautés d’utilisateurs qui sont interconnectées et peuvent 
partager des contenus instantanément. Ils regroupent autant des individus que des institutions 
et entreprises grâce à leurs multiples fonctionnalités. Les bibliothèques vont ainsi s’en servir 
pour promouvoir leurs ressources numériques qui sont tous types de documents ou logiciels qui
n’ont pas de support physique, qui est relié à un appareil électronique.
Maintenant, ce sujet pose plusieurs questions : quels sont les réseaux sociaux les plus 
pertinents pour la promotion des ressources numériques, l’impact que cette utilisation a sur le 
métier des bibliothécaires ou encore quelles sont les techniques utilisées par les bibliothèques 
sur les RSN ? Pour traiter cette problématique, nous allons dans un premier temps vous proposer 
un état de l’art global de l’évolution du sujet dans le temps. D’abord, en essayant de relever les 
réseaux sociaux qu’emploient le plus les bibliothèques. Ensuite, comment elles vont les 
utiliser. Et enfin, l’aspect de leur utilisation pour la promotion des ressources numérique.\newpage

\section{État de l'art} % section 2
Les bibliothèques universitaires (BU) ont dû adapter leurs méthodes de communication. 
Effectivement, l’époque a évolué, on est actuellement dans ce qu’on appelle le digital age. Les 
générations évoluent de plus en plus avec la technologie dès le plus jeune âge. C’est en 2005, 
d’ailleurs, qu’apparait le terme « bibliothèque 2.0 »  par Casey \citet{kouakou2015determinants,}.
La promotion des ressources et services des bibliothèques universitaires s’est vue ainsi évoluée. En effet, avec le développement des technologies de l’information et de la communication, les habitudes des personnes en matière d’accès à l’information sont affectées. Le public préfère trouver des informations en utilisant les médias sociaux et internet \citet{rahmawati2021academic,}.C’est pourquoi les bibliothèques se sont vues contraintes à s’adapter à cette 
nouvelle approche de communication pour répondre aux besoins de ses utilisateurs. Pour les 
bibliothèques universitaires, il s’agissait « d’être là où les publics sont » \citet{marois2012reseaux,}.Le choix du réseau 
social pour la bibliothèque universitaire devient donc un élément crucial pour s’assurer 
d’atteindre la cible recherchée.\\
Lorsque les sites de réseautage de médias sociaux, tels que YouTube, Flickr, Twitter et 
Facebook, ont fait leur apparition, leurs caractéristiques ont attiré les bibliothèques, car elles 
leur permettent de faire participer les utilisateurs, ainsi que de promouvoir leurs informations 
et leurs services. En plus, les RSN « permettent aux informations de toucher un public bien 
plus large que celui initialement visé par la bibliothèque » \citet{leclercq2011valorisation,} Dès lors, de plus en plus de 
bibliothèques ont inclus les réseaux sociaux comme principaux outils de communication avec 
leurs communautés d’utilisateur en dehors des murs physiques de la bibliothèque6. Harrison et 
al., en 2017, ont montré que l’utilisation des réseaux sociaux par les bibliothèques est devenue très importante durant les 15 dernières années \citet{harrison2017social,}. En 2013, Chu et Du expliquent comment les RSN 
sont utilisés par les bibliothèques universitaires pour promouvoir leurs services, interagir avec 
les étudiants et améliorer la communication entre les bibliothécaires et leurs utilisateurs \citet{chu2013social,}. Selon 
Ramsey et Vecchione, les RSN favorisent l’engagement, la créativité et la collaboration \citet{ramsey2014engageant,}

\subsection{ Quels réseaux sociaux et pourquoi ?}

Le nombre de réseaux sociaux ne fait qu'augmenter. En 2016, une étude a recensé 260 médias sociaux \citep{mesguich2017bibliothèques,}. Si on regarde en 2022, l’enquête menée par Overdrive a dénombré pas moins de 675 réseaux sociaux \citep{jin2022social,}. Parmi cette large panoplie de réseaux sociaux, certains d’entre eux sont les plus retrouvés dans les stratégies de promotion des bibliothèques universitaires. Maintenant, la question qui se pose est quels types de réseaux sociaux les bibliothèques investissent?

\subsubsection{Facebook}

Une étude a été faite aux États-Unis en 2010 portant sur 100 bibliothèques et sur ce total, 89 utilisaient Facebook et 81 Twitter \citep{palmer2014caractérisant,}. En 2011, une autre enquête menée auprès des 21 bibliothèques membres du Conseil des bibliothèques universitaires de l'Ontario a révélé que 62 pourcent utilisaient Twitter et 52 pourcent utilisaient Facebook. \citep{collins2012social,} Ristiyono Mohammad Pandu a également démontré que c’était Twitter et Facebook les outils de médias sociaux les plus populaires parmi la plupart des bibliothèques du monde \citep{ristiyono2021effectiveness,}. Les publications sur l’utilisation des sites de réseautages sociale, comme Twitter et Facebook, sont abondantes. Ils sont parfois évoqués comme RSN incontournables, «ils ont pour points communs d’être à la fois généralistes, informatifs et interactifs» \citep{mesguich2017bibliotheques,}. Mais ils ne fonctionnent pas de la même manière. Facebook est une plateforme qui permet des interactions avec les utilisateurs. Il n’empêche qu’Aharony a constaté que les bibliothèques n’utilisaient pas ce RSN comme plateforme de discussion, mais elles l’emploient pour fournir des informations \citep{aharony2012facebook,}.Comme l’explique Véronique Mesguish :« L’objectif de création d’une page Facebook ne peut se limiter à la volonté d’être présent “là où sont ses lecteurs”, ni à la diffusion d’informations pratiques ou institutionnelles. N’oublions pas les principes essentiels de ce réseau : la timeline, le renouvellement continu, l échange, le partage, la communauté». \citep{mesguich2017bibliotheques,}
\\Plusieurs études ont prouvé que les bibliothèques utilisent Facebook pour : présenter les nouveautés, la promotion des ressources numériques, promotion de leur site internet, organiser des événements, service de chat, question/réponse \citep{deaudureau,}. Facebook est aussi employé pour la création de la communauté et pour fournir des liens statiques vers des ressources statiques de la bibliothèque \citep{palmer2014characterizing,}. La page Facebook permet de diffuser des informations, mais aussi de lier le site internet de la bibliothèque ou encore de mettre des informations sur la bibliothèque comme l’adresse mail, le numéro de téléphone, les horaires par exemple \citep{fernandez2016redes,}. Toutes ces fonctionnalités ne sont pas forcément propres à ce réseau social, mais le fait qu’elle soit rassemblée aux mêmes 
endroits permette une facilité d’utilisation pour les utilisateurs. \citep{stylianou2015review,}

\subsubsection{Instagram}

Néanmoins, plusieurs études ont été menées pour examiner l’utilisation que les bibliothèques font d’autres réseaux sociaux. Notamment sur Instagram qui est une plateforme différente de Facebook et Twitter, c’est un réseau social axé sur l'image. Comme l’explique Rachman « The primary purpose of using Instagram is to share visual content, and it is used as 
a fun learning tool as one can post interesting pictures with accompanying information or facts
» \citep{rachman2021examinant,}. Le résultat de son étude menée en 2019 sur 15 bibliothèques universitaires démontre
qu’elles utilisent Instagram généralement pour partager des informations à propos des services 
proposés, de l’actualité de la bibliothèque, d’informer sur les nouveaux documents et de 
communiquer avec les utilisateurs \citep{himes1954value,}.\\
Ce qu’Instagram propose n’est pas nouveau, mais la différence c’est que le plus important est le visuel. Comme dans l’article de Rahmawati et Rahmi qui analyse des résultats d’études menées sur l’utilisation des réseaux sociaux en bibliothèque universitaire, ils notent qu’Instagram et Facebook seraient les plateformes dominantes en BU en ce qui concerne la promotion à travers le format de contenus photos et vidéos \citep{rahmawati2021academic,}. Cette conclusion viendrait confirmer le positionnement du réseau Instagram dans la stratégie des bibliothèques universitaires.

\subsubsection{Twitter}
Twitter est un média social de microblogging, «
c'est-à-dire une forme de publication très courte (…), publiée en temps réel et accessible à tout un chacun » \citep{mesguich2017bibliotheques,}. Twitter est utilisé pour communiquer avec des individus et pour se tenir au courant des informations sur les nouvelles ressources ou les évènements à venir \citep{cordier2013diététique,}. Une enquête a été menée par Sultan M. Al-Daihani et Suha A. AlAwadhi sur l’utilisation de Twitter par les bibliothèques universitaires qui explique que les « libraries utilize Twitter effectively for users to view library activities, collections and services in a focused and clear manner» \citep{al2015explorer,}. C’est un RSN qui nécessite l’immédiateté de l’information et ainsi oblige la bibliothèque à être plus attentif aux utilisateurs et plus actives25.
\\Hugot Christophe a résumé l’utilisation de Twitter par les bibliothèques universitaires (BU)comme : « complément au panneau d’affichage, à la tête de gondole, à l’organe de presse et de publicité, au standard » \citep{ruiz2016redes,}. Il y a eu beaucoup de recherche sur l’utilisation de Twitter ou de Facebook par les bibliothèques en anglais, mais notamment en espagnol comme celle de Dicac Margaix-Arnal sur l’utilisation de Facebook \citep{margaix2008bibliotecas,} ou celle de Rafael Carrasco-Polaino et al. qui a mené une analyse de l’utilisation de Twitter en bibliothèque universitaire espagnole. \citep{carrasco2019redes,} 

\subsubsection{Youtube}
Un dernier réseau social qu’il est intéressant d’aborder dans ce travail, c’est YouTube. Selon Tan et Pearce, 188 titres d’articles contiennent le mot YouTube en 2012 \citep{cho2013youtube,}. Les bibliothèques ont envisagé également de créer une présence dessus. Il est intéressant d’étudier ce canal parce qu’il utilise comme support principal la vidéo et qu’il regorge de ressources numériques. Cependant, les bibliothèques françaises n’ont pas encore beaucoup investi ce 
terrain comparé à leurs homologues américaines \citep{cho2013youtube,}.\\
Effectivement, Allan Cho a attesté qu’en 2007, de grandes universités américaines avaient amorcé une intégration de YouTube pour diffuser des contenus multimédias. Par exemple, l’Université de Californie à Berkeley ou encore de Colombia a mis en ligne des cours et des conférences sur la plateforme. L’université de Yale a également décidé d’enregistrer ses professeurs pour les poster comme a fait le MIT.\citep{cho2013youtube,} Malheureusement, peu d’articles abordent l’utilisation de YouTube en bibliothèque universitaire, principalement pour la promotion de leur catalogue numérique. Cependant, plusieurs études ont démontré que les bibliothèques utilisaient majoritairement YouTube pour faire des tutoriels. Majid et al., ont constaté que 
beaucoup de BU emploient la plateforme YouTube pour « for teaching information literacy skills ». \citep{majid2012analysis,}

\subsubsection{Réseaux sociaux littéraires}

Il existe aussi des réseaux sociaux numériques littéraires comme Babelio, Goodread, 
LibraryThing, etc. Ce sont des RSN qui sont spécialisés dans les livres et la lecture créée vers les années 2000. Louis Wiart explique que : «[l]e développement des réseaux sociaux littéraires est pensé par ces acteurs comme un moyen de fédérer des communautés autour de leurs marques et d’associer les internautes à la valorisation de leurs activités ».\citep{wiart2019presence,} Ce sont des plateformes qui peuvent aussi intéresser les bibliothèques, «
ils apportent aux bibliothèques des outils nouveaux pour la recommandation des ouvrages, pour la classification en ligne ainsi que pour l’établissement et l’entretien de communautés de lecteurs » \citep{wiart2019presence,} \\
Même si ces plateformes n’ont pas été créées dans un premier temps pour les bibliothèques. Ce sont des réseaux qui permettent aussi de posséder une identité numérique pour une institution et donc qui sont progressivement investis par les bibliothèques. Surtout, que certaines plateformes comme Babelio ou LibraryThing offrent aux bibliothèques un enrichissement « de leurs catalogues informatisés avec des contenus produits par les lecteurs » \citep{wiart2019presence,}. Ces enrichissements sont construits pour appuyer les bibliothécaires dans leur rôle de conseiller aux lecteurs. Ainsi, ça remet en question l’entièreté du rôle de bibliothécaire. Malgré le fait que ces RSN permettent un lien avec une communauté de lecteur, il n’y a aucune recherche concrète qui a été menée sur le lien entre les BU et ce type de réseau social numérique.\\
On peut remarquer que les meilleurs réseaux sociaux pour la promotion de tous les types de ressources numériques sont de tout type et de tout genre. Dès lors, les bibliothèques ont besoin d’être multitâches pour être partout en même temps. Comme l’explique Palmer : « any value created for an organization through social media comes not from any particular plateforms, but from how they are used » \citep{palmer2014characterizing,}

\section{Bibliothécaires et gestion de contenus}

Bien sûr, peu importe le canal de communication utilisé, il est important pour les bibliothèques d’avoir un rythme constant. Twitter et Instagram nécessitent un rythme de publication très régulière. C’est une plateforme orientée vers l’actualité, l’information en temps réel, donc cela réclame une cadence importante \citep{mesguich2021chapitre,}. En plus, il est important de ne pas utiliser tous ces réseaux de communication de la même manière. Comme l’explique Louis Wiart : «
Le recours à l’image et à des textes brefs, la quotidienneté des publications, tout comme la propension à la légèreté et à l’humour, témoignent d’une appropriation des codes d’expression et d’animation de communauté propres aux réseaux sociaux » \citep{bar1960r,}. L’équipe chargée des réseaux sociaux pour la promotion de Gallica explique également qu’il faut «
l’élaboration d’une ligne éditoriale […] adaptable en fonction des réseaux sur lesquels elle s’exprime » \citep{leroy2012bibliotheque,}. Malheureusement, la difficulté que rencontrent les bibliothèques universitaires est que « même si elle s’adresse prioritairement à la communauté universitaire, cette dernière est très hétérogène » \citep{hugot2017albert,}. De ce fait, la problématique rédactionnelle se soulève. C’est le cas par exemple du SCD de Caen qui explique que comme elle doit suivre un public hybride, elle doit renouvelerconstamment ses formes d’interaction et créer de nouveaux projets pour essayer de garder l’attention de ses utilisateurs.\citep{chuiton2018construire,}\\

Il existe également certains codes qu’il faut connaitre pour avoir plus de visibilité sur les RSN, par exemple il faut utiliser les hashtags. Un hashtag ou mot-dièse est « un marqueur de métadonnées, qui se présente sous la forme d’un mot, d’un groupe de mots, voire d’une phrase précédée du signe » \citep{mesguich2017bibliotheques,}. Il faut que celui-ci concerne le sujet de la publication, comme le précise Mesguish Véronique : « grâce aux hashtags, les abonnés pourront suivre les différentes publications sur un sujet donné » .\citep{mesguich2017bibliotheques,} Cet outil peut aussi s’utiliser sur la plupart des réseaux sociaux.\\
Tous ces outils réclament dès lors des connaissances, ce qui soulève une question : qui 
gère la .communication sur les réseaux sociaux? Est-ce que ce sont des bibliothécaires ou des spécialistes
? C’est une problématique que lève Diallo et Haquet : « [l]a révolution numérique crée ainsi de nouvelles pratiques professionnelles, et par là contribue à remodeler l’identité professionnelle des bibliothécaires » \citep{mesguich2017bibliotheques,}. De ce fait, comme le souligne Louis Wiart: « L’engagement des bibliothèques sur les réseaux sociaux ne va donc pas de soi : il se traduit par des arbitrages et des confrontations entre des prises d’initiative et des contraintes inhérentes aux plateformes mobilisées, à la situation concrète des établissements et aux rapports engagés avec les publics » \citep{wiart2019presence,}\\

En conclusion, les RSN nécessitent des compétences, des savoir-faire que les bibliothécaires ont besoin d’apprendre. Comme l’explique Véronique Mesguish, il y a plusieurs types de compétences qui rentrent en ligne de compte : techniques, rédactionnelles, relationnelles et juridiques 59. Nicolas Malais l’explique très bien que le manque de formation des bibliothécaires peut freiner l’évolution des bibliothèques \citep{di2018produire,}. Dans le cas de l’Indonésie, selon des études, l’une des principales difficultés à la promotion en bibliothèques universitaires sur les médias sociaux serait le manque de spécialisation des bibliothécaires dans le domaine des réseaux sociaux numériques61. Une autre recherche sur les BU sudafricaines a aussi souligné que la mauvaise connexion à internet et le manque de formation, les empêchaient d’utiliser correctement les canaux de communication \citep{rabatseta2021adoption,}\\

Cependant, à l’heure actuelle, comme l’expliquent Slouma et Chevry-Pebayle « La question est […] de savoir pour une institution comment maîtriser sa présence, son identité et sa réputation sur ces médias sociaux » \citep{slouma2018presence,}. Ainsi, avant tout de chose, les bibliothèques doivent construire leurs identités numériques. Pour ce faire, elles ont deux possibilités comme l’explique Dujol Lionel, elles peuvent choisir une approche institutionnelle ou une approche par connaissance \citep{di2018produire,}. Ça veut dire que soit «[leurs] contenus arborent l’identité institutionnelle incarnée par le logo et le nom de la bibliothèque» \citep{stylianou2015review,} soit les contenus sont organisés par thème. Par exemple, la Bibliothèque nationale de France a expliqué en 2012, avoir choisi de créer une page spéciale pour leur catalogue numérique Gallica. De ce fait, la BnF a plusieurs pages Facebook, Twitter, etc.\citep{stylianou2015review,}. Elle a opté pour une identité numérique par thème.

\subsection{Les ressources numériques}

Les ressources numériques se sont imposées de plus en plus dans les bibliothèques 
universitaires. Cependant, comme l’explique Iriate et al. : 
« La maintenance d’une collection numérique en expansion constante est complexe, la 
nature instable des médias électroniques (formats, liens, etc.) et la multiplicité des modèles 
de services (licences, plateformes, Open Access, etc.) engendrent un bouleversement des 
pratiques professionnelles dans le monde des bibliothèques académiques » \citep{iriarte2018ebooks,}.
Les bibliothèques se sont donc transformées, leurs ressources sont à la fois sur papier ou 
en numérique, donc les bibliothèques physiques et numériques sont intrinsèquement liées. \citep{iriarte2018ebooks,}\\
Ensuite, une grande question pour les bibliothèques, se pose quand on parle des ressources numériques, c’est qu’elles sont les avantages que les ressources apportent et Aurélie Vieux souligne que : « Les avantages principaux des ressources en ligne par rapport à leur double papier sont qu’ils sont faciles d’accès, peuvent être consultés à tout moment depuis n’importe où, qu’ils sont téléchargeables et transportables, qu’ils peuvent être utilisés par plusieurs personnes à 
la fois et qu’ils accélèrent la recherche par mots-clés grâce à la recherche plein texte» \citep{vieux2014signaler,}. 
Son étude a aussi révélé plusieurs problèmes, soit les ressources électroniques ne sont pas connues des utilisateurs dus à leur manque de visibilité comme elles n’ont pas de support physique. Il est vrai qu’elles font partie de la part invisible du web, on ne les trouve pas dans les moteurs de recherches comme l’explique Véronique Mesguish \citep{mesguich2017bibliothèques,}. Soit, quand les ressources sont connues des utilisateurs, ils disent que parfois les plateformes ne sont pas intuitives, donc elles sont difficiles d’utilisation \citep{vieux2014signaler,}. Comme l’explique Isabelle Westeel « Les internautes ne commencent presque jamais leurs recherches sur les sites des bibliothèques patiemment et savamment construits par les bibliothécaires»\citep{mesguich2017bibliothèques,}.
\\Dans la promotion de ces ressources, les BU utilisent de ce fait plusieurs moyens. Dans un premier temps, leur site internet, les emails, les newsletters. Frédéric Souchon a même fait son mémoire sur la valorisation des ressources numériques en bibliothèque physique. Il explique certains cas où les bibliothèques ont utilisé un système de QR code pour attirer 
l’attention des utilisateurs en bibliothèque sur les ressources électroniques\citep{souchon2014faire,}. Il existe de plus le 
cas de la bibliothèque universitaire de Genève qui a créé une application pour promouvoir ces ressources électroniques \citep{iriarte2018ebooks,}. Mais très vite les bibliothécaires se sont rendu compte que pour visibiliser leurs ressources ils devaient être là où le public est. Dès lors, les bibliothèques ont utilisé les réseaux sociaux pour promouvoir leurs ressources numériques. D’ailleurs, Mesguish Véronique souligne que : « Facebook et Twitter se prêtent particulièrement bien à la promotion des nouvelles acquisitions d’e-books, ou la mise en avant de titres en relation avec des sujets 
d’actualité pour les lecteurs »\citep{mesguich2017bibliothèques,}
\\Par exemple, la Bibliothèque nationale de France a expliqué en 2012, avait choisi de créer une page spéciale pour leur catalogue numérique Gallica. De ce fait, la BnF a plusieurs pages Facebook, Twitter, etc.\citep{leroy2012bibliotheque,} Elle a opté pour une identité numérique par thème. L’équipe chargée des réseaux sociaux pour la promotion de Gallica explique également qu’il faut « l’élaboration d’une ligne éditoriale […] adaptable en fonction des réseaux sur lesquels elle s’exprime »\citep{leroy2012bibliotheque,}. Ensuite, pour chaque RSN, elle communique d’une façon précise, la BnF ne crée pas un post de la même manière pour Facebook que pour Twitter. Il faut s’approprier les réseaux sociaux. Comme l’explique Louis Wiart : « Le recours à l’image et à des textes brefs, la quotidienneté des publications, tout comme la propension à la légèreté et à l’humour, témoignent d’une appropriation des codes d’expression et d’animation de communauté propres aux réseaux sociaux »\citep{leroy2012bibliotheque,}.
\\La majorité des sources qui existent à l’heure actuelle sont des études de cas sur la promotion des services et ressources des bibliothèques en général. Très rapidement, on se rend compte que beaucoup d’études sont consacrées, tout d’abord à comment introduire les médias sociaux dans le plan marketing d’une bibliothèque universitaire. Ensuite, des études sur
l’utilisation desréseaux sociaux en général en bibliothèque universitaire et enfin sur l’utilisation d’un canal de communication spécifique en BU.
\\En bibliothèque universitaire, nous avons le cas des bibliothèques de l’université de 
Caen Normandie qui en février 2013 ont fait leur migration sur les réseaux sociaux \citep{chuiton2018construire,}. Le processus s’est fait en plusieurs étapes, d’abord sur le réseau Facebook à travers un groupe dédié à la bibliothèque. Ensuite, ils ont décidé d’ouvrir des comptes sur Twitter en fonction des différentes disciplines couvertes par les différentes bibliothèques, leur objectif étant de valoriser les collections, mais aussi l’action, les services de la BU, la valorisation de l’open access, les archives ouvertes, la fiabilité de l’information et l’innovation en bibliothèque. En 2014, ils ont ajouté à leur ensemble de réseaux sociaux Flickr, SlideShare, YouTube et Pinterest. La bibliothèque utilise chaque canal de communication en fonction des particularités de chacun.
\\Toutefois, très peu de recherches ont été menées sur l’utilisation des réseaux sociaux dans le but de promouvoir les ressources électroniques des bibliothèques universitaires en particulier. Il existe des études de cas sur la promotion des ressources numériques à travers les réseaux sociaux en bibliothèque nationale (exemple : Bnf) ou en bibliothèque publique (la 
Bibliothèque de l’École polytechnique fédérale de Lausanne (EPFL)\citep{wagnieres2012etude,}. De ce fait, dans la suite de ce travail, on va vous présenter en détail trois études de cas en bibliothèque universitaire sur la promotion des ressources numérique sur les réseaux sociaux dans le monde.

\subsection{bibliothèque universitaire de Grenade} 
La bibliothèque universitaire de Grenade (BUG) dans le sud de l’Espagne a décidé d’inclure les réseaux sociaux dans leur stratégie de communication. Comme ils le disent : « La moderna comunicación no puede concebirse sin ellas
» \citep{fernandez2016redes,}. C’est un moyen de communication bidirectionnel entre la bibliothèque et ces utilisateurs, la BUG les utilise pour informer leurs utilisateurs des services et ressources qui existent, mais aussi pour observer leurs réactions. Tout d’abord, la bibliothèque avait convenu pendant l’année universitaire, 2013-2014, de n’utiliser que Facebook comme réseau social, mais pour finir elle a aussi créé un compte Pinterest, Twitter et plus tard elle a fait son apparition également sur YouTube et Instagram. On va dans cette étude de cas analyser d’abords quels réseaux sociaux la bibliothèque utilise et comment elle a organisé leurs utilisation. Ensuite, cette enquête va essayer de chercher
comment la BUG les utilise pour promouvoir les ressources numériques.

\subsubsection{ Quels réseaux sociaux ?}

Dans un premier temps, la bibliothèque a créé une page Facebook avec pour motivation 
la journée internationale du livre. Ensuite, la bibliothèque a décidé, en 2015, d’accroitre sa 
présence en créant un compte Twitter. Elle a choisi de se mettre sur ce réseau pour toucher plus 
les étudiants. Ils expliquent : « con Twitter buscábamos acercarnos más a los estudiantes, tanto 
para recibir información como para facilitar las consultas a la biblioteca y además esta red 
parecía el medio más extendido entre los jóvenes ».\citep{lagarde2019renvoi,}En 2016, c’était le canal le plus utiliser 
par les usagers pour demander des renseignements ou exposer des problèmes et des critiques.\citep{lagarde2019renvoi,} 
Dès lors, le groupe charger des réseaux sociaux a dû également apprendre à gérer et rediriger les commentaires négatifs.

De plus, la bibliothèque s’est créé une présence numérique sur Pinterest. Ce canal a été proposé par le groupe responsable du patrimoine de la bibliothèque dans le but de promouvoir la collection des documents anciens sur les RSN.\citep{fernandez2016redes,} Par conséquent, la bibliothèque a principalement commencé à utiliser Pinterest pour la promotion des documents numérisés, « El tablero de pruebas comenzó centrado en promocionar las exposiciones recogidas en el Portal BiblioTesoros, con dos vertientes, dar acceso a los documentos digitalizados y accesibles a través del Repositorio Institucional Digibug y, en caso de no estar digitalizado, dar acceso al propio Portal »\citep{fernandez2016redes,}. Le contenu des publications est axé sur la diffusion d’information liée à la bibliothèque : diffusion des services, diffusion des activités qu’organise la bibliothèque, diffusion des contenus électroniques, diffusion de la culture. Pinterest permet aussi de rendre plus visible le site web de la bibliothèque surtout, c’est l’image de celle-ci. Comme l’expliquent ceux chargés des RSN de la bibliothèque:« El uso de Pinterest está dando visibilidad y promocionado la imagen de la Biblioteca y que sus contenidos ayudan aumentar el tráfico a la página web de la Biblioteca así como al Repositorio Institucional ».\citep{fernandez2016redes,}

\subsection{La bibliothèque universitaire de Laval }

\subsubsection{ Présentation de l’université}

L’université de Laval est une université canadienne située dans la ville du Québec. Elle 
offre un large programme de formation, pour différents cycles. Elle dispose pour ses étudiants 
d’une bibliothèque regorgeant de grandes quantités de ressources physiques et numériques. En 
2015 du résultat des comptes de ses ressources, la Bibliothèque Universitaire de Laval (BUL) 
donnait accès à 4 219 809 documents physiques, et à 280 886 documents numériques, dont 212 
123 livres électroniques et 68 763 périodiques. Parmi ses nombreuses facultés, nous avons la 
Faculté des sciences de l’agriculture et de l’alimentation (FSAA) pour qui la bibliothèque 
universitaire de Laval a mis en œuvre ce projet pilote de “mise en valeur des nouveautés en 
format électronique sur Pinterest” lancé le 24 septembre 2013. Projet qui vient répondre à un 
besoin pour la Bibliothèque universitaire de Laval de donner plus de visibilité à ses collections 
en explorant un nouveau moyen de communication qui est le réseau social numérique (RSN) 
Pinterest.

\subsubsection{ Présentation de la stratégie mise en place et déroulement du projet}

Le projet pilote “Pinterest pour la FSAA” ciblait principalement les étudiants de la Faculté 
des Sciences de l’Agriculture et de l’Alimentation. L’objectif de ce projet était de promouvoir 
et de rendre accessible les livres en format électroniques pour ces étudiants qui dû à leur 
formation étaient appelés à beaucoup travailler en extérieur et ceux-ci nécessitaient donc un 
accès rapide et fluide aux ressources de la bibliothèque, également d’être informé des nouvelles 
parutions dans le catalogue de la bibliothèque.

Le projet a démarré par une phase pilote qui avait pour but de récolter les avis du public 
cible et analyser le comportement face à ce nouvel outil. La phase test a duré 03 mois, du 24 
septembre 2013 à fin décembre de la même année. Les porteurs du projet se sont appuyés sur 
des évènements physiques pour promouvoir le nouvel outil Pinterest, avec des sessions de test 
séances tenantes pour initier les participants à l’utilisation du réseau. En plus de cela ils ont eu 
recours à l’affichage au sein de la bibliothèque, l’insertion des liens de redirection du compte 
Pinterest sur tous les portails (site web) de la faculté et de l’université.

\subsubsection{ Pourquoi Pinterest ?}

Le choix du réseau social dans une stratégie de promotion est un élément primordial, 
celui-ci n’est pas un choix aléatoire, car il doit répondre au profil et aux habitudes de la cible. 
Il s’agit « d’être là où les publiques sont » \citep{marois2012reseaux,}. Dans le cas de la BUL, le réseau Pinterest est 
idéal car les statistiques nous montrent que les usagers de la bibliothèque s’y trouvent.
\\La BUL est la première bibliothèque universitaire francophone au Canada à utiliser le 
réseau social Pinterest pour la promotion de sa collection et de ses services. Le choix de ce 
réseaux social comme moyen de promotion par la Bibliothèque de Laval s’est fait 
principalement pour sa popularité, car elle se classait 3ème auprès des réseaux Facebook et 
Twitter et grâce à son taux d’utilisateurs au sein de la communauté canadienne, qui se classait 
au deuxième rang avec un total de 3,6 pourcent de tous les utilisateurs et de 3,3 pourcent des visites parmi 
45 pourcent de nouveaux utilisateurs enregistrés en juin 2013. (Semiocast, 2013). Comme le défini Daniela Zavala-Mora, Pinterest est un « tableau d’affichage » virtuel qui permet d’organiser et de partager des images et des vidéos, liées à des URL\citep{locatorsuniform,}, provenant d’internet et portant sur différents sujets.
\\Les avantages de Pinterest pour une bibliothèque universitaire sont nombreux, il permet de 
promouvoir les livres électroniques et imprimés sur différents sujets, de créer de l’interaction 
entre la bibliothèque et ses usagers, à travers les commentaires sur les contenus apportés aux 
différents tableaux du compte Pinterest, le partage des tableaux ce qui donne une grande 
visibilité à la collection de la bibliothèque, un accès rapide et facile à l’application sur différents 
appareil, la gestion de contenus , l’ajout d’images plutôt aisé.

\subsubsection{ Utilisation de l’outil Pinterest dans le cadre du projet de la BUL}

L’utilisation du réseaux social Pinterest démarre tout d’abord par la création d’un 
compte, comme pour tout réseaux social, et cela est gratuit. Après avoir créé le compte vous 
pouvez donc effectuer toutes les actions que le réseau met à votre disposition, création de 
tableaux thématiques ( ici comparables à des rayons de bibliothèques), ajout des contenus 
images et vidéos en fonction de vos tableaux, l’abonnement à des tableaux d’autres utilisateurs, 
vous pouvez aussi partager vos propres tableaux pour plus de visibilité. Pinterest est caractérisé 
ici comme un réseau ouvert car il favorise le partage de contenus sur d’autres réseaux comme 
Facebook, twitter, par email…

\subsubsection{ Suivi et évaluation}

A la suite de l’implémentation de l’outil Pinterest pour la bibliothèque universitaire de 
Laval, il était primordial comme pour tout projet d’évaluer les résultats de cette activité. Pour 
cela, les gestionnaires du compte de la BUL sur Pinterest utilisent Pinterest Analytics qui est 
l’outil d’analyse gratuit qu’offre Pinterest à ses utilisateurs ayant un compte professionnel. L’objectif ici étant de ressortir les statistiques d’utilisation : abonnements, partages, commentaires et interactions, comme sur tout réseaux social numérique. Ainsi de façon hebdomadaire ils pouvaient ressortir :
\begin{itemize}
\item Les informations sur l’audience : leurs données démographiques et centres d’intérêts et 
le nombre de personnes qui visitent le compte
\item Le nombre de fois où les images du compte ont été affiché dans le profil d’un utilisateur 
pour donner suite à une recherche à partir de mots clés appelés ici “impressions”
\item Le nombre de fois où les images des tableaux ont été sauvegardé par un utilisateur 
“réépinglés”
\item Le nombres de clics sur les images du tableau pour accéder au contenu intégral de la 
ressource
\end{itemize}

\subsubsection{ Observation des évolutions mises en place}

Suite à une analyse de l’évolution de l’activité sur le compte de la bibliothèque universitaire de Laval de 2013 (date de création du compte) à nos jours, une légère progression est observée sur le nombre d’abonnés qui est passé de 106 à 406. La question qu’on pourrait se poser à la vue de ces chiffres est de savoir pourquoi l’évolution est légère pour un écart de 9 
ans. L’animation du compte est-elle toujours régulière comme indiqué dans le planning au démarrage du projet, vue le nombre de ressources actuelles sur le compte. En rappel en 2014 après la phase test du projet le compte comptait 1086 épingles et 165 abonnés au compte.

\subsection{Bibliothèque universitaire de Jinan}

\subsubsection{ Observation des évolutions mises en place}

Avec le développement des nouvelles technologies médiatiques, les moyens de service des bibliothèques universitaires subissent également d’énormes changements. Les usagers des bibliothèques universitaires utilisent de plus en plus les réseaux sociaux pour avoir accès auxressources numériques pour leurs études, leurs recherches et leurs apprentissages \citep{harrison2017social,}.C’est dans cette optique que Qiandong Zhu à travers son article « The application of social media in outreachof academic libraries' resources' and services : A case study on WeChat », décrit ce qui a été fait dans la bibliothèque de l’université de Jinan en chine.

\subsubsection{Bref Présentations de la bibliothèque de l'université de Jinan et les réseaux sociaux utilisés}

\subsubsection{ Présentations}

Fondée en 1948 sous le nom de Jinan Union University , située à Guangzhou en Chine,jinan est l'une des plus vielles universités de Chine et l'une des «211 » universités nationales clés en Chine. La bibliothèque occupe une superficie de 63 000 mètres carrés et est équipée d’un système numérique, d’un système de recherche d’informations, avec une collection de plus de 4,6 millions de livres imprimés et électroniques, et 4 200 types de périodiques chinois et 
étrangers, et possède 23 000 types de revues électroniques \citep{koopman2007world,}.

\subsubsection{ Les réseaux sociaux utilisés}

Les bibliothèques universitaires chinoise utilisent plusieurs réseaux sociaux populaires qui ont les fonctionnalités similaires avec d’autres réseaux sociaux utilisé à l’international. Comme le dit Qiandong Zhu « Renren and Q-zone can be deemed as a Chinese equivalent of Facebook. Similarly, Sina Weibo and Tencent Weibo are the Chinese equivalent of Twitter. WeChat evolved from QQ, a dominant instant message tool in china » \citep{application zhu2016,} \newpage

\section{Conclusion} % section 4

En conclusion, on remarque, dans un premier temps, autant dans l’université de Grenade que celle de Laval ou dans la plupart des articles étudiés, que ce sont les bibliothécaires, organisées en groupe, qui sont chargées des réseaux sociaux numériques. De ce fait, le problème qui a déjà été remarqué par certains auteurs comme Nicolas Malais120 et qui se remarque aussi dans les études de cas analysés, c’est le manque de formation des bibliothécaires. Ils ne savent 
pas forcément les codes qu’il faut utiliser. Ensuite, même quand ils les connaissent, ils sont 
parfois bloqués, car leur public est très hétéroclite. 
\\ Après, on aperçoit que quand les BU commencent à créer leur présence sur les RSN, elles ont en général choisi d’abord un seul canal. Dans le cas de la BUG, c’était Facebook, pour Laval c’était Pinterest et enfin pour Jinan, Weibo. Après, les bibliothèques universitaires ont commencé à s’étendre sur d’autres médias sociaux, elles n’utilisent jamais qu’un seul canal de communication. On peut peut-être l’expliquer par le fait que tous ces réseaux sociaux se complètent entre eux, Twitter est celui qu’il faut utiliser pour le partage direct d’une information, Instagram complète Facebook dans le sens où les utilisateurs vont plus faire attention à l’image publiée tandis que sur Facebook les gens vont être attirés par le texte. 
Pinterest lui est employé comme une exposition virtuelle qui permet de s’organiser en thématique. YouTube quant à lui est plus axé sur des tutoriels. De ce fait, les BU ont besoin d’utiliser tous ces outils, car ils n’attirent pas le même type de public. Pour le cas de Jinan, cela fonctionne de la même façon. La bibliothèque avait besoin d’outil que Weibo ne leur offrait pas donc ils ont étendu leur présence numérique sur WeChat. 
\\Enfin, quand on regarde précisément comment les bibliothèques universitaires 
valorisent leurs ressources numériques, on remarque plusieurs points de vue. Soit la 
bibliothèque de Grenade qui veut être partout en même temps, mais qui ne fait rien de particulier 
pour la valorisation des ressources électroniques, soit l’université de Laval qui a organisé un 
projet spécial autour de la promotion de ceux-ci comme l’université de Jinan qui a créé un 
compte WeChat spécialement pour rendre visible leurs ressources numériques.

\bibliographystyle{plainnat} % paramètre l'affichage de la bibliographie
\bibliography{biblio2} % indique que la bibliographie se trouve dans le fichier biblio.bib


\end{document} % fin du corps du texte